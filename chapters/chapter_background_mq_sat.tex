\chapter{{SAT Problem}}
\label{chap:satisfiability}

In our thesis we studied the security of multivariate cryptography from the perspective of solving instances of the Boolean Satisfiability Problem, which is also known as the SAT problem. 

%Thus, before this study we focused on the concepts of the SAT problem starting by defining itself. After, we will describe a special case of the SAT problem which is widely used in multivariate cryptography.

The SAT problem is defined as follows. First, one is given a logical expression in the five operators of predicate calculus $(\wedge, \vee, \neg, \Rightarrow, \iff )$ but not the existential or universal quantifiers ($\exists, \forall$). Then, one is asked if the logical expression has a assignment for each of its variables, such that the logical expression evaluates to true. Unlike, the problem if the expression has the value false, for every one its possible assignments, is called UNSAT.\basedon{\cite[p. 200]{bard2009algebraic}}

The solving of the SAT problem has been extensively studied along of years and many methods to solve it have been developed. The most efficient methods seem to be all derived from the method proposed by \cite{davis1958reductions}. Their proposal uses as input a logical expression in Conjunctive Normal Norm (CNF). A formula is in CNF form if it is a conjunction of clauses, and each clause is a disjunction of literals. 

In follow, we define these concepts more formally.

\begin{defn}
The language of Boolean formulae consists of Boolean variables, whose values are True or False; Boolean operators such as negation $(not)$, conjunction $(\wedge)$, disjunction $(\vee)$, implication $(\Rightarrow)$, equivalence $(\iff)$; and parentheses.
\end{defn}

\begin{defn}
A literal is either an atomic formula or the negation of an atomic formula.
\end{defn}

\begin{defn}
A clause is a disjunction of $n >0$ literals
\end{defn}
The most study and promises SAT solvers are based on the work of Davis and Putnam. In this thesis, for the cryptanalysis against multivariate quadratic scheme we use a kind of solvers called Clause Driven Clause Learn SAT (CDCL-SAT). As we will see, these SAT solver are inspired on the Davis and Putnam method and uses as input a logical expression in CNF. 

%https://www8.cs.umu.se/kurser/TDBAfl/VT06/algorithms/BOOK/BOOK3/NODE112.HTM
When the maximum number of literals presented in each clause of a CNF logical expression is $k$, then the respective SAT problem is known as $k$-CNF-SAT problem. Since clause sets with only one literal per clause are easy to satisfy, the computer scientist, and particularly in this thesis, we are interested in slightly larger classes. Exactly what is the clause size at which the problem turns from polynomial to hard? This transition occurs when each clause contains three literals, the so-called $3$-CNF-SAT problem. Follow there is a formalization of the $3$-CNF-SAT problem.\\
\\
\noindent
\textbf{3-CNF-SAT Problem.} Let $B=\{b_1,\cdots,b_n\}$ be a set of Boolean variables, let $L=\left\{b_1,\overline{b_1},\cdots,b_n,\overline{b_n}\right\}$ be the corresponding set of literals, let $c_i\in \left(L\cup L^2\cup L^3\right)$ be clauses of at most 3 literals, and let $C=\left\{ c_1, \cdots, c_m\right\}$ be a set of these clauses. Then the corresponding 3-CNF-SAT problem is to determine if there is an assignment $A\in \left\{0,1\right\}^n$ for $B$ such that all $c_i$ are true and hence $C$ is satisfied.\\
\\
\noindent
The worse-case estimates for solving SAT on a CNF problem are commonly given in terms of the numbers of clauses $c$, number of variables $v$, the total length of all clauses $L$, and the Strahler number of the tree-like resolution proof. In section we will review several methods for trying to formalize the hardness of solving SAT problems.
%Thus, logical expression encoding as 3-CNF-SAT problems are most efficient to solve when the number of variables $n$ is used a measure of difficulty of the $k$-CNF-SAT problems. Thus, in this thesis we encoding our multivariate quadratic systems as 3-CNF-SAT problems.



%\textbf{SAT-Problem.} Give a Boolean formula $\Phi$ of $n$ variables and $m$ clauses. Is there any model for $\Phi$?\\
%\\
%\noindent
